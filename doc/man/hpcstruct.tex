%% $Id$

%%%%%%%%%%%%%%%%%%%%%%%%%%%%%%%%%%%%%%%%%%%%%%%%%%%%%%%%%%%%%%%%%%%%%%%%%%%%%
%%%%%%%%%%%%%%%%%%%%%%%%%%%%%%%%%%%%%%%%%%%%%%%%%%%%%%%%%%%%%%%%%%%%%%%%%%%%%

\documentclass[english]{article}
\usepackage[latin1]{inputenc}
\usepackage{babel}
\usepackage{verbatim}

%% do we have the `hyperref package?
\IfFileExists{hyperref.sty}{
   \usepackage[bookmarksopen,bookmarksnumbered]{hyperref}
}{}

%% do we have the `fancyhdr' or `fancyheadings' package?
\IfFileExists{fancyhdr.sty}{
\usepackage[fancyhdr]{latex2man}
}{
\IfFileExists{fancyheadings.sty}{
\usepackage[fancy]{latex2man}
}{
\usepackage[nofancy]{latex2man}
\message{no fancyhdr or fancyheadings package present, discard it}
}}

%% do we have the `rcsinfo' package?
\IfFileExists{rcsinfo.sty}{
\usepackage[nofancy]{rcsinfo}
\rcsInfo $Id$
\setDate{\rcsInfoLongDate}
}{
\setDate{2020/07/15}
\message{package rcsinfo not present, discard it}
}

\setVersionWord{Version:}  %%% that's the default, no need to set it.
\setVersion{=PACKAGE_VERSION=}

%%%%%%%%%%%%%%%%%%%%%%%%%%%%%%%%%%%%%%%%%%%%%%%%%%%%%%%%%%%%%%%%%%%%%%%%%%%%%
%%%%%%%%%%%%%%%%%%%%%%%%%%%%%%%%%%%%%%%%%%%%%%%%%%%%%%%%%%%%%%%%%%%%%%%%%%%%%

\begin{document}

\begin{Name}{1}{hpcstruct}{The HPCToolkit Performance Tools}{The HPCToolkit Performance Tools}{hpcstruct:\\ Recovery of Static Program Structure}

\Prog{hpcstruct} - recovers the static program structure of a CPU or GPU binary, including procedures, inlined functions, loops, and source lines

See \HTMLhref{hpctoolkit.html}{\Cmd{hpctoolkit}{1}} for an overview of \textbf{HPCToolkit}.

\end{Name}

%%%%%%%%%%%%%%%%%%%%%%%%%%%%%%%%%%%%%%%%%%%%%%%%%%%%%%%%%%%%%%%%%%
\section{Synopsis}

\Prog{hpcstruct} \oOpt{options} \Arg{binary}

\Prog{hpcstruct} \oOpt{options} \Arg{measurement directory of GPU-accelerated application}

%%%%%%%%%%%%%%%%%%%%%%%%%%%%%%%%%%%%%%%%%%%%%%%%%%%%%%%%%%%%%%%%%%
\section{Description}

Given an application binary, a shared library, or a GPU binary, \Prog{hpcstruct}
recovers the program structure of its object code by analyzing available information
about loop nests, inlined functions, and the mapping between machine instructions and 
source lines.  Program structure is a
mapping of a program's object code to its static source-level structure.

When analyzing a CPU binary \Arg{b}, by default, 
\Prog{hpcstruct} writes its results to the file 'basename(\Arg{b}).hpcstruct'.
To improve attribution of performance measurements to program source code, one can
pass one or more program structure files (e.g., for an executable and/or 
one or more shared libraries) to HPCToolkit's analysis tool \Prog{hpcprof}
along with one or more HPCToolkit performance measurement directories.

During execution of a GPU-accelerated application on an NVIDIA GPU, HPCToolkit
records NVIDIA 'cubin' GPU binaries in the application's measurement directory.
To attribute performance to GPU functions in a GPU-accelerated application, one
should apply \Prog{hpcstruct} to the application's HPCToolkit measurement directory to
analyze all GPU binaries recorded within. When analyzing a measurement directory
that includes GPU binaries, any program structure files produced will be recorded
inside the measurement directory. When \Prog{hpcprof} is applied to a measurement
directory that contains program structure files for GPU binaries, these program
structure files will be used to help attribute any GPU performance measurements.

\Prog{hpcstruct} is designed primarily for highly optimized binaries created from
C, C++, Fortran, and CUDA source code. Because \Prog{hpcstruct}'s algorithms exploit a
binary's debugging information, for best results, binary should be compiled
with standard debugging information or, at a minimum, line map information.
Note: although a CPU or GPU binary may be optimized, it must also be
compiled with line mappings for the results to be useful, e.g., normally using the -g
flag when compiling CPU binaries. 

For faster analysis of large binaries or many GPU binaries, we recommend using
the -j option to employ multithreading. As many as 32 cores can be used profitably
to analyze large CPU or GPU binaries in the measurements directory for a
GPU-accelerated application.


%%%%%%%%%%%%%%%%%%%%%%%%%%%%%%%%%%%%%%%%%%%%%%%%%%%%%%%%%%%%%%%%%%
\section{Arguments}

%% SKW 7/6/18:
%% 
%% The following arguments are implemented in hpcstruct's "Args.cpp" but are deprecated
%% and so not documented here, per JohnMC
%% 
%%    agent-c++
%%    agent-cilk
%%    loop-intvl
%%    loop-fwd-subst
%%    normalize
%%    use-binutils

\begin{Description}
\item[\Arg{binary}] File containing an executable or dynamically-linked shared library.
Note that \Prog{hpcstruct} does not recover program structure for libraries that \Arg{binary} depends on.
To recover that structure, run \Prog{hpcstruct} on each dynamically-linked shared library
or relink your program with static versions of the libraries.

\item[\Arg{measurement directory of GPU-accelerated application}] 
A measurement directory of a GPU-accelerated application that employed NVIDIA GPUs.
When a GPU-accelerated application runs on an NVIDIA GPU, its 'cubin' GPU binaries are recorded into 
HPCToolkit's measurement directory.
Applying \Prog{hpcstruct} to a measurement directory analyzes any GPU binaries recorded 
in the measurement directory during execution. 

\end{Description}

Default values for an option's optional arguments are shown in \{\}.

\subsection{Options: Informational}

\begin{Description}

\item[\Opt{-V}, \Opt{--version}]
Print version information.

\item[\Opt{-h}, \Opt{--help}]
Print help message.

\end{Description}

\subsection{Options: Parallel Usage}
\item[\OptArg{-j}{num}, \OptArg{--jobs}{num}]
Use \Arg{num} threads for all phases in \Prog{hpcstruct}. {1}

\item[\OptArg{--gpu-size}{n}]
Size (bytes) of a GPU binary that will cause \Prog{hpcstruct}
to use \Arg{num} threads to analyze a binary in parallel.
GPU binaries with fewer than \Arg{n} bytes will be analyzed
concurrently, \Arg{num} at a time.  {100000000}

\subsection{Options: Structure recovery}

\begin{Description}

% No longer exist.
%
% \item[\OptArg{--demangle-library}{libpath}]
% Use a function from the library at \emph{libpath} to demangle C++ names.
% By default the library function used is \Prog{__cxa_demangle}.
% A different function may be specified with the \Prog{--demangle-function} option.
% 
% \item[\OptArg{--demangle-function}{funcname}]
% Call the function named \emph{funcname} in the specified demangler library to demangle C++ names.
% This option may only be given if the \Prog{--demangle-library} option is also given.

\item[\OptArg{--gpucfg}{yes/no}]
Compute loop nesting structure for GPU machine code.  Currently,
this applies only to NVIDIA CUDA binaries (cubins). Loop nesting
structure is only useful with instruction-level measurements
collected using PC sampling. \{no\}

\item[\OptArg{-I}{path}, \OptArg{--include}{path}] 
Use \Arg{path} when resolving source file names. 
This option is useful when a compiler records the same filename in different ways within the symbolic information.
(Yes, this does happen.)
For a recursive search, append a '+' after the last slash, e.g., \texttt{/mypath/+}. 
This option may appear multiple times.

\item[\OptArg{-R}{'old-path=new-path'}, \OptArg{--replace-path}{'old-path=new-path'}]
Replace instances of \Arg{old-path} with \Arg{new-path} in all paths with \Arg{old-path} is a prefix
(e.g., a profile's load map and source code).
Use \verb+'\'+ to escape instances of '=' within specified paths.
This option may appear multiple times.
  
Use this when a profile or executable references files that have been relocated,
such as might occur with a file system change.

\end{Description}

\subsection{Options: Output}

\begin{Description}

\item[\OptArg{-o}{file}, \OptArg{--output}{file}]
Write results to \Arg{file}.  \{\Arg{basename(binary)}\File{.hpcstruct}\}

% \item[\Opt{--compact}]
% Generate compact output by eliminating extra white space.

\end{Description}

\subsection{Options for Developers:}

\begin{Description}
\item[\OptArg{--jobs-struct}{num}]
Use num threads for the program structure analysis phase of \Prog{hpcstruct}.

\item[\OptArg{--jobs-parse}{num}]
Use num threads for the parse phase of \Prog{hpcstruct}.

\item[\OptArg{--jobs-symtab}{num}]
Use num threads for the symbol table analysis phase of \Prog{hpcstruct}.

\item[\Opt{--show-gaps}]
Developer option to
write a text file describing all the "gaps" found by \Prog{hpcstruct},
i.e. address regions not identified as belonging to a code or data segment
by the ParseAPI parser used to analyze application executables.
The file is named \emph{outfile}\File{.gaps}, which by default is
\emph{appname}\File{.hpcstruct.gaps}.

\item[\Opt{--time}]
Display the time and space usage per phase in \Prog{hpcstruct}.

\end{Description}



%%%%%%%%%%%%%%%%%%%%%%%%%%%%%%%%%%%%%%%%%%%%%%%%%%%%%%%%%%%%%%%%%%
\section{Examples}

\begin{enumerate}
\item 
Assume we have used HPCToolkit to collect performance measurements for the (optimized) CPU binary 
\File{sweep3d} and that performance measurement data for the application is in the measurement 
directory \File{hpctoolkit-sweep3d-measurements}. 
Assume that \File{sweep3d} was compiled with debugging information using the -g compiler flag in addition to any
optimization flags. 
We wish to recover program structure in \File{sweep3d}
for use with \HTMLhref{hpcprof.html}{\Cmd{hpcprof}{1}}.
To do this, execute:

\begin{verbatim}
    hpcstruct sweep3d
\end{verbatim}

By default, the output is placed in a file named \File{sweep3d.hpcstruct}. 

To use the program structure file to help interpret performance measurements in \File{hpctoolkit-sweep3d-measurements}, 
provide the program structure file to \Prog{hpcprof} using the -S option, as shown below:

\begin{verbatim}
    hpcprof -S sweep3d.hpcstruct hpctoolkit-sweep3d-measurements
\end{verbatim}

Additional program structure files for any shared libraries used by \Prog{sweep3d} 
can be passed to \Prog{hpcprof} using additional -S options. 

\item
Assume we have used HPCToolkit to collect performance measurements for the (optimized) GPU-accelerated 
CPU binary \File{laghos}, which offloaded computation onto one or more NVIDIA GPUs.
Assume that performance measurement data for the application is in the measurement 
directory \File{hpctoolkit-laghos-measurements}. 


Assume that the CPU code for \File{laghos} was compiled with debugging information using the -g compiler flag in addition to any
optimization flags and that the GPU code the application contains was compiled with line map information (-lineinfo).

To recover program structure information for the laghost CPU binary, execute:

\begin{verbatim}
    hpcstruct laghos
\end{verbatim}

By default, the output is placed in a file named \File{laghos.hpcstruct}.

To recover program structure information for the GPU binaries used by laghos, execute:

\begin{verbatim}
    hpcstruct hpctoolkit-laghos-measurements
\end{verbatim}

The measurement directory will be augmented with program structure information recovered for the GPU binaries.

As shown below, one can provide the program structure files to \Prog{hpcprof} by passing the 
CPU program structure file explicitly using the -S argument and passing the program structure files for the GPU binaries
implicitly by passing the measurement directory. 

\begin{verbatim}
    hpcprof -S laghos.hpcstruct hpctoolkit-laghos-measurements
\end{verbatim}

\end{enumerate}

%%%%%%%%%%%%%%%%%%%%%%%%%%%%%%%%%%%%%%%%%%%%%%%%%%%%%%%%%%%%%%%%%%
\section{Notes}

\begin{enumerate}

\item For best results, an application binary should be compiled with debugging information.
To generate debugging information while also enabling optimizations,
use the appropriate variant of \verb+-g+ for the following compilers:
\begin{itemize}
\item GNU compilers: \verb+-g+
\item Intel compilers: \verb+-g -debug inline_debug_info+
\item IBM compilers: \verb+-g -fstandalone-debug -qfulldebug -qfullpath+
\item PGI compilers: \verb+-gopt+
\item NVIDIA's nvcc: \\
~~~~\verb+-lineinfo+ provides line mappings for optimized or unoptimized code\\
~~~~\verb+-G+ provides line mappings and inline information for unoptimized code
\end{itemize}

\item While \Prog{hpcstruct} attempts to guard against inaccurate debugging information,
some compilers (notably PGI's) often generate invalid and inconsistent debugging information.
Garbage in; garbage out.

\item C++ mangling is compiler specific. On non-GNU platforms, \Prog{hpcstruct}
tries both platform's and GNU's demangler.

\end{enumerate}

%%%%%%%%%%%%%%%%%%%%%%%%%%%%%%%%%%%%%%%%%%%%%%%%%%%%%%%%%%%%%%%%%%
%% \section{Bugs}
%% 
%% \begin{enumerate}

%% \item xxxxxx

%% \end{enumerate}

%%%%%%%%%%%%%%%%%%%%%%%%%%%%%%%%%%%%%%%%%%%%%%%%%%%%%%%%%%%%%%%%%%
\section{See Also}

\HTMLhref{hpctoolkit.html}{\Cmd{hpctoolkit}{1}}.

%%%%%%%%%%%%%%%%%%%%%%%%%%%%%%%%%%%%%%%%%%%%%%%%%%%%%%%%%%%%%%%%%%
\section{Version}

Version: \Version

%%%%%%%%%%%%%%%%%%%%%%%%%%%%%%%%%%%%%%%%%%%%%%%%%%%%%%%%%%%%%%%%%%
\section{License and Copyright}

\begin{description}
\item[Copyright] \copyright\ 2002-2021, Rice University.
\item[License] See \File{README.License}.
\end{description}

%%%%%%%%%%%%%%%%%%%%%%%%%%%%%%%%%%%%%%%%%%%%%%%%%%%%%%%%%%%%%%%%%%
\section{Authors}

\noindent
Rice University's HPCToolkit Research Group \\
Email: \Email{hpctoolkit-forum =at= rice.edu} \\
WWW: \URL{http://hpctoolkit.org}.

\LatexManEnd

\end{document}

%% Local Variables:
%% eval: (add-hook 'write-file-hooks 'time-stamp)
%% time-stamp-start: "setDate{ "
%% time-stamp-format: "%:y/%02m/%02d"
%% time-stamp-end: "}\n"
%% time-stamp-line-limit: 50
%% End:

